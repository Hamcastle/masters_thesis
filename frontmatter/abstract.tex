%!TEX root = ../dissertation.tex
% the abstract

\section*{Introduction}
Diseases that damage the center of the retina produce both central blind spots (scotomas) and disrupt oculomotor control. Such disruptions limit individuals' ability to read or identify faces. In response, many  develop a preferred retinal locus (PRL) -- effectively a ``pseudo-fovea'' that restores some level of pre-disease state functional vision and oculomotor control. This process is not well understood, as it is difficult to use eye tracking systems to study the effects of PRL development on oculomotor control in individuals with real retinal disease.

However, a growing literature suggests that it is possible to model this process in individuals with healthy retinas using a gaze-contingent simulated scotoma to produce a ``pseudo-PRL''(referred to in this work as a ``pPRL''). The studies presented in this work capitalize on these findings by modeling the process of oculomotor accommodation to the development of a central scotoma in individuals with healthy retinas using an experimentally induced pPRL at different locations and eccentricities relative to the center of vision. We also investigated the effects of dynamic performance feedback on the stability and rate of formation of the resulting pPRL, as well as the degree to which this learning was retained over time in the absence of training.

\section*{Methods}
In the first two studies, subjects completed two sessions of fixational stability training separated by approximately a week. In the third, they received only a single session of the same training. In the first experiment, a training session contained 200 training trials; in the second, 100 trials; and in the third, 50 trials.

During a training trial, subjects centered a gaze-contingent ring over a stable fixation target and held their gaze on that location, as best as they were able, for fifteen seconds. If they remained ``on target'' in this way for one and a half seconds, the diameter of the ring was reduced, making the task more difficult. If instead they instead fell ``off target'' during the same period, the diameter of the ring was increased, making the task easier. Subjects were therefore instructed to ``make the ring as small as possible.''

The position and orientation of the gaze contingent ring relative to the foveated point of regard (FPOR) was adjusted between experiments, but was always kept a minimum of 6.4\degree of visual angle eccentric to it. To ensure that subjects could not use their fovea to help complete the task, the fixation target was removed from the display if the FPOR fell within a 2\degree radius of the target's center. This effectively simulated an absolute, circular 2\degree central scotoma.

All three studies shared this core method. However, each contained a single manipulation to a parameter which was otherwise held constant in the other two. In Experiment 1, the ring was positioned either 6.4\degree to the right or below the FPOR. This location was switched between training sessions; the order in which the two conditions were completed was counter-balanced between subjects. In Experiment 2, the ring was always to the right of the FPOR, but the distance between it and the ring was adjusted between training sessions to be either 6.4\degree or 12.8\degree. In Experiment 3, subjects completed only a single training session where they either received feedback from the gaze-contingent ring or did not.

Subjects' performance in all three studies was assessed in terms of their gaze ``accuracy'' (the distance between the centroid of their recorded gaze points and the center of the fixation target) and ``stability'' (the ``tightness'' of the cluster of recorded gaze points around the centroid). Retention of learning in terms of both performance measures between training sessions was also assessed in Experiments 1 and 2.

\section*{Results}
Both fixational stability and accuracy were significantly enhanced at pPRL locations following the two sessions of training. In  both Experiment 1 and 2, only a main effect of training session number was observed, suggesting both that there is no intrinsic advantage or disadvantage to pPRL formation at either of the tested orientations and eccentricities at baseline, or that beginning training at either conferred an additional advantage or disadvantage during subsequent training. Performance at the beginning of the second training session was also significantly better than at the beginning of the first, suggesting that task learning was retained across time. Finally, a significant effect of feedback was observed in Experiment 3, suggesting that the performance gains observed in the first two experiments were not simply attributable to a process of task learning over time.  

\section*{Discussion}
Results from these studies are encouraging: the proposed training method improves fixational stability and accuracy at trained pPRL locations. Experiment three also suggests that the method is successfully capitalizing on the known benefits of perceptual feedback for other types of tasks, and thus accelerates performance gains relative to a baseline rate of learning. Future studies will investigate whether the observed performance gains lead to enhanced functional vision at the trained site.
