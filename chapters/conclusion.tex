%!TEX root = ../dissertation.tex
\chapter{Conclusion}
\label{conclusion}

In this thesis, I presented three experiments that tested two types of parameters associated with a novel method of fixational stability and accuracy training: those associated with the visual system of the trainee (orientation and eccentricity of the trained site relative to the fovea) and one associated with the design of the method itself (provision of real-time performance driven feedback). All three experiments demonstrated that this method  significantly improved the positional accuracy of subjects' fixations relative to stable targets; in two out of the three, fixational stability with respect to the selected location also significantly improved. Results from Experiments 1 and 2 suggested that the orientation and eccentricity of the trained site relative to the fovea does not significantly impair performance gains associated with this method, and that state changes within each parameter do not reduce or eliminate these benefits. Effects of training also appear to survive extinction as a function time without training. Finally, I concluded on the basis of results from Experiment 3 that these improvements are not solely attributable to some type of task-specific learning, but are rather driven to a large extent by the feedback provided by the training system.

These results together paint an interesting and exciting picture for the practical applicability and success of this method. More specifically, they suggest both that the method is effective in stabilizing gaze and improving the accuracy of fixation location selection relative to targets, and that the location of and distance between the trained pPRL site and the former fovea does not interfere with or impair subjects' ability to reap these benefits. Of more general scientific interest is a potential inference to be drawn from these findings: though there is undoubtedly an oculomotor control map centered on the fovea, the shape and parameters of this map may be shaped or redrawn to a far more flexible degree than is likely for functional visual performance. This is because functional vision is to a large extent constrained by the physiology of the retina: it is not known, but is unlikely, that retinal visual parameters such as receptor field sizes can be manipulated as a function of visual performance feedback. Certainly others such as distributions of photoreceptor cell-types are unlikely to be plastic in this way.

This basic-scientific implication of my thesis entails one further for the practical application of our training method. Specifically, if the site of the pPRL in the oculomotor control map can be shifted at will through training without penalizing oculomotor performance, then it can be used to induce a pPRL at locations chosen to maximize likely functional performance based on knowledge of the spatial distribution of this feature of the visual system. Presumably the resulting stabilization of gaze at this location would therefore enable trainees to maximize their functional vision at that site. What is more, both our data and the logic of our conclusions imply that the effects of our method in this regard are flexible and robust enough to endure large, sudden changes in pPRL location and eccentricity relative to the fovea. Given that many retinal disease that lead to PRL formation are degenerative, that our method is powerful in this way greatly elevates its likelihood for potential use in clinical settings. 

The chief limitation of the work presented in this document is that it still remains a \textit{simulation} of a disease process, and one which explicitly does not model certain features of true PRL (such as the ``unconscious'' selection mechanism by which they appear to emerge) and may have entirely missed others. At least superficially, the validity of all of the above claims with regard to the \textit{practical} application of this approach are therefore contingent on testing with actual patients. This transition is complicated both by those features of retinal disease that make simulations necessary in this area, and by a relatively limited fundamental understanding of what a PRL \textit{actually is}. At least one peer-reviewed paper \cite{crossland_2011} has attempted to address this issue simply by polling researchers studying this phenomenon and compiling a preliminary ``consensus definition'' using qualitative methods. Surveyed researchers were nearly exactly split on whether a PRL could be ``defined in a subject with an artificial scotoma'' (and thus whether a pPRL really models a true PRL). However, if I were ultimately to produce evidence showing that equivalent results to those in the presented studies could be replicated among subjects with actual retinal disease, this would have a further basic-scientific benefit of helping to establish an equivalence between pPRL and true PRL.

A further issue concerns whether adjusting fixational stability at locations is actually a helpful process in terms of functional vision. Several papers have shown, for example, that there may be an ``ideal'' temporal sampling rate of the visual system at different locations on the retina\citep{deruaz_2004, watson_2012}. This rate is derived at least in part from the natural tremor movements of the eye as a way of both improving its sampling efficiency \textit{and} preventing fading at crucial sites on the retina. To the extent that our method would push the jitter rate below some critical threshold at a particular location, it might induce unwanted fading, preventing subjects from seeing critical objects of interest. However, this literature is fairly small, and even if the basic model that underlies it is correct, it should still be possible to adapt our approach to simply reach the optimum level of stability in order to leave an acceptable amount ``jitter'' in place to prevent fading.

I have also only investigated training effects at only two orientations and two eccentricities relative to the fovea: below and to the right, and to the right at eccentricities of 6.4\degree and 11.2\degree. Although it is assumed that meridional effects on fixational stability are symmetric along axes and at different eccentricities, it is possible that this is not actually the case. Finally, I have also only examined the effect of our training program in terms of performance in the simplest oculomotor control scenario: maintaining a stable fixation on a highly-visible, non-moving, eccentrically positioned target. It is unknown whether they will extend to more complicated oculomotor tasks such as making saccades to randomly re-positioned targets or reading. Further studies are currently being developed to test our method under such conditions.

