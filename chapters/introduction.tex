%!TEX root = ../dissertation.tex
\chapter{Introduction}
\label{introduction}

Individuals with retinal disease that damages or destroys the central, high-resolution portion of the visual field (the fovea) may develop one or more preferred retinal loci (PRL). \citep{vonnoorden_1962,whittaker_1988,fletcher_1997,white_1990,whittaker_1991}. A PRL is effectively a ``pseudo-fovea'', and restores a degree of pre-insult visual performance. Those who develop PRL typically do so over a series of weeks to months, and tend to use either a single general-purpose \citep{vonnoorden_1962,whittaker_1988,fletcher_1997} or a small set of task-specific PRL \citep{timberlake_1986}.

A number of studies have demonstrated that the formation of a PRL is associated with improved functional vision performance \citep{palmer_2010,crossland_2004,seiple_2005}. Low-vision rehabilitation specialists therefore often train PRL in patients with pathological central vision loss \citep{schuchard_2005,watson_2006}. However, very few of these studies have specifically examined changes in oculomotor control associated with the acquisition of a PRL \citep{crossland_2004}. Understanding this process is important as visual tasks such or reading and face recognition require precise eye movement control. Loss of such may therefore  impair visual performance beyond the perceptual effects of the disease alone. This claim is supported by evidence suggesting that fovea loss directly limits oculomotor control \citep{schuchard_2005,crossland_2004,bullimore_1995}, and that individuals with central field loss perform many visual tasks less well than healthy controls, even while using undamaged retinal regions at equivalent distances and orientations relative to the fovea \citep{timberlake_1986,mcmahon_1991,mcmahon_1993}.

One reason for the limited research in this area may be features of retinal disease itself. Relevant patient populations tend to express quite variable symptom profiles, making obtaining sufficiently homogeneous samples of research subjects challenging \citep{bowers_1997}. Fixational instability that accompanies fovea loss can also make the calibration of eye tracking systems difficult or even impossible. Much of what \textit{is} known therefore comes from experiments where central scotomas are simulated for individuals with healthy vision. These methods use eye tracking systems to present gaze contingent stimuli on computer displays that reflect the ``experience'' of retinal disease \citep{mcilreavy_2012,aguilar_2011}. These simulations have significant technical limitations but, if carefully controlled, provide useful first-order approximations of the perceptual and behavioral effects of such diseases \citep{bowers_1997}. They also facilitate statistically powerful within-subjects/cross-over experimental designs, which are generally not possible with patients.

A small but growing literature shows that over the course of several hours of explicit training with simulated central vision loss, individuals with healthy retinas can form a ``PRL-like'' region in the periphery of their retina that exhibits many of the known adaptive features of real PRL-formation \citep{varsori_2004,kwon_2013,walsh_2014}. We call such regions``pseudo-PRL'' (pPRL) because although they appear to be functionally similar to real PRL, they are not a product of a disease processes and may therefore still differ from them in important and as-yet understood ways. Nevertheless, these results are encouraging, as they suggest it is possible to study changes in oculomotor control associated with PRL formation without the difficulties of studying them ``in the wild''.  We therefore used a pPRL induction paradigm derived from these studies to address the following questions:

\begin{itemize}
\item Does pPRL development affect fixational stability or the magnitude of oculomotor errors around a target? Such changes are thought to be one of the key features of oculomotor adaptation to retinal disease \citep{crossland_2004} and are considered an important objective in rehabilitation programs as well \citep{mandelcorn_2013}. To our knowledge, however, there is little available data on changes to oculomotor control as \textit{a function of} the implicit time-course of training or practice at a PRL or pPRL site (though again see \cite{walsh_2014,kwon_2013,varsori_2004} for some discussion of this issue).

\item The performance of the visual system is known to vary as a function of orientation and eccentricity relative to the fovea. Do these meridional asymmetries in visual function also impact oculomotor changes associated with pPRL development? It is also widely understood that visual performance falls as a function of increasing retinal eccentricity. Will pPRL development and stability ultimately be affected by the eccentricity of the training site relative to the fovea?

\item Is pPRL training retained across time and transferred across locations? Many retinal diseases progress over time and therefore a trained PRL may eventually be claimed by an advancing lesion \citep{nilsson_1998}. We therefore ask whether the development of a pPRL at one location influences the course of pPRL redevelopment at a subsequent trained location.

\item If it is in fact possible to train a pPRL, then can this training process be accelerated by providing trainees with real-time performance feedback?
\end{itemize}

We attempted to address these questions through three related experiments. In each, a single parameter was manipulated: either the pPRLs eccentricity from or orientation relative to the fovea, or the presence of absence of real time feedback. All three experiments are given their own chapter in this document. Each begins with a brief review of literature focused tightly around the literature related to the particular parameter manipulated in that experiment. As they shared a common set of methods and outcome measures, and because the data were analyzed in largely the same way, a detailed description of the methods is provided only in the Methods section of the chapter for the first experiment. Results are summarized and briefly discussed in each chapter, the larger significance of all the reported findings is discussed in a separate Discussion chapter at the end of the document. 
